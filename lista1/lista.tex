\documentclass{article}
\usepackage[brazil]{babel}
\usepackage[utf8]{inputenc}
\usepackage{url}
\usepackage{multicol}
\usepackage[bottom=2cm,top=2cm,left=2cm,right=2cm]{geometry}
\usepackage{graphicx}
\usepackage{float}
\usepackage{caption}
\usepackage[dvipsnames]{xcolor}
\usepackage{tikz}
%\usepackage{subcaption}

\usepackage{amssymb}
\usepackage{setspace}
\usepackage{indentfirst}
\usepackage{mathtools}
\usepackage{amsmath}
\usepackage{subfigure}

\title{Lista 1\\
    \large Sistemas Baseados em Conhecimento: \texttt{[MAC0444]}\\}

\author{Julia Leite\\
    \large Nusp: 11221797}

\begin{document}
    
\maketitle

Resolução

\begin{enumerate}
    
    \item Para cada uma das três sentenças abaixo, encontre uma interpretação que faça a sentença
    falsa e as outras duas verdadeiras:

    \begin{itemize}
        \item [a)] $S = \forall x \forall y \forall z ((P (x,y) \wedge P(y,z)) \implies P(x,z))$
        
            \begin{list}{-}{Interpretações}

                \item $\Im_1 = <A, I>$ e $\Im_1 \models S$   \texttt{[Verdadeiro]}
                
                    $A = \{0,1,2\}$

                    $P(x,y) = \{(0,0),(1,1),(2,2)\}$\\

                \item $\Im_2 = <A, I>$ e $\Im_2 \models S$   \texttt{[Verdadeiro]}
                
                $A = \{0,1,2,3\}$

                $P(x,y) = \{(0,0),(1,1),(2,2),(3,3)\}$\\

                \item $\Im_3 = <A, I>$ e $\Im_3 \models S$   \texttt{[Falso]}
                
                $A = \{0,1,2\}$

                $P(x,y) = \{(0,1),(0,2),(1,0),(1,2),(2,0),(2,1)\}$\\
                
            \end{list}

        \item [b)] $\forall x \forall y ((P (x,y) \wedge P(y,x)) \implies x=y)$
        
        \begin{list}{-}{Interpretações}
            \item $\Im_1 = <A, I>$ e $\Im_1 \models S$   \texttt{[Verdadeiro]}
            
                $A = \{1,2\}$

                $P(x,y) = \{(1,1), (2,2)\}$\\

            \item $\Im_2 = <A, I>$ e $\Im_2 \models S$   \texttt{[Verdadeiro]}
            
            $A = \{1,2\}$

            $P(x,y) = \{(1,1), (1,2), (2,2)\}$\\

            \item $\Im_3 = <A, I>$ e $\Im_3 \models S$   \texttt{[Falso]}
            
            $A = \{1,2\}$

            $P(x,y) = \{(1,1), (1,2), (2,1), (2,2)\}$\\
            
        \end{list}
        
        \item [c)] $\forall x \forall y ((P (a,y) \implies P(x,b))$
        
        \begin{list}{-}{Interpretações}
            \item $\Im_1 = <A, I>$ e $\Im_1 \models S$      \texttt{[Verdadeiro]}
            
                $A = \{a,b\}$

                $P(x,y) = \{(a,a), (a,b), (b,b)\}$\\

            \item $\Im_2 = <A, I>$ e $\Im_2 \models S$      \texttt{[Verdadeiro]}
            
            $A = \{a,b,c\}$

            $P(x,y) = \{(a,a), (a,b), (a,c), (b,b), (c,b)\}$\\

            \item $\Im_3 = <A, I>$ e $\Im_3 \models S$   \texttt{[Falso]}
            
            $A = \{a,b\}$

            $P(x,y) = \{(a,a), (a,b)\}$\\
            
        \end{list}

    \end{itemize}

    \item Antônio, Maria e João são membros do Clube Alpino. Todo membro do Clube Alpino que
    não é esquiador é um alpinista. Alpinistas não gostam de chuva, e qualquer um que não
    goste de neve não é esquiador. Maria não gosta de nada que Antônio gosta, e gosta de
    qualquer coisa de que Antônio não gosta. Antônio gosta de chuva e de neve.

    \begin{itemize}
        \item [a)] Observação: nesse item, abreviamos Clube Alpino para CA.
        
            Obtivemos o seguinte conhecimento (KB)

            \begin{itemize}
                \item [] $CA (Antonio)$
                \item [] $CA (Maria)$
                \item [] $CA (Joao)$
                \item [] $\forall x (CA(x) \land \neg Esquiador(x) \implies Alpinista(x))$
                \item [] $\forall x (Alpinista(x) \implies \neg Gosta (x, chuva))$
                \item [] $\forall x (\neg Gosta (x, neve) \implies \neg Esquiador (x))$
                \item [] $\forall x (Gosta (Antonio, x) \implies \neg Gosta (Maria, x))$
                \item [] $\forall x (\neg Gosta (Antonio, x) \implies Gosta (Maria, x))$
                \item [] $Gosta(Antonio, chuva)$
                \item [] $Gosta(Antonio, neve)$
            \end{itemize}
        
        \item [b)] Queremos saber se:
        
            $$KB \vDash \exists x (CA(x) \land Alpinista(x))$$

            Sabemos que $KB \vDash Gosta(Antonio, neve)$ e $KB \vDash \forall x (Gosta (Antonio, x) \implies \neg Gosta (Maria, x))$
            então, $$KB \vDash \neg Gosta(Maria, neve)$$

            Como $KB \vDash \neg Gosta(Maria, neve), \forall x (\neg Gosta (x, neve) \implies \neg Esquiador (x))$, temos que
            $$KB \vDash \neg Esquiador(Maria)$$.

            Sabemos que $KB \vDash CA (Maria), \neg Esquiador(Maria)$ e $KB \vDash \forall x (CA(x) \land \neg Esquiador(x) \implies Alpinista(x))$,
            então: $$KB \vDash Alpinista(Maria)$$.

            Sabemos, então, que a existência de um membro do CA que é 
            Alpinista é consequência semântica do conhecimento

        \item [c)] Base de conhecimento (KB) em outra notação:
        
            \begin{itemize}
                \item [I] $[CA (Antonio)]$
                \item [II] $[CA (Maria)]$
                \item [III] $[CA (Joao)]$
                \item [IV] $[\neg CA(x), Esquiador(x), Alpinista(x)]$
                \item [V] $[\neg Alpinista(x), \neg Gosta(x, chuva)]$
                \item [VI] $[Gosta(x, neve), \neg Esquiador (x)]$
                \item [VII] $[\neg Gosta (Antonio,x), \neg Gosta (Maria,x)]$
                \item [VIII] $[Gosta (Antonio,x), Gosta (Maria,x)]$
                \item [IX] $[Gosta(Antonio, chuva)]$
                \item [X] $[Gosta(Antonio, neve)]$

            \end{itemize}
        
        Vamos tentar provar $\alpha$ que sem a informação de que \textit{Maria não gosta de nada de que Antônio gosta},
        ou seja sem o item VII

        $$\alpha = \exists x (CA(x) \land Alpinista(x)$$
        $$\neg \alpha = \forall x (\neg CA(x) \lor \neg Alpinista(x)$$
        $$\neg \alpha = [\neg CA(x), \neg Alpinista(x)]$$

        Seja nossa base de conhecimento sem o item VII: KB', queremos provar que
        $KB' \vDash \alpha$, ou seja que, $KB' \cup \{\neg \alpha\}$ não é 
        satisfatível.

        Então temos:

        $\neg \alpha = [\neg CA(x), \neg Alpinista(x)]$ e $[\neg CA(x), Esquiador(x), Alpinista(x)]$ (IV)
        $$[\neg CA(x), Esquiador(x)]$$

        $[\neg CA(x), Esquiador(x)]$ e $[Gosta(x, neve), \neg Esquiador (x)]$ (VI)

        $$[\neg CA(x), Gosta(x, neve)]$$

        Agora: $[\neg CA(x), Gosta(x, neve)]$ e $[CA (Antonio)]$ (I):

        $$\left[\frac{x}{Antonio}\right][Gosta(Antonio, neve)] \neq []$$

        Mesmo se tentarmos substituir o passo anterior com outros itens, 
        não chegamos em []:

        Outra tentativa: $[\neg CA(x), Gosta(x, neve)]$ e $[CA (Maria)]$ (II):

        $$\left[\frac{x}{Maria}\right][Gosta(Maria, neve)] \neq []$$

        Ou: $[\neg CA(x), Gosta(x, neve)]$ e $[CA (Joao)]$ (III):

        $$\left[\frac{x}{Joao}\right][Gosta(Joao, neve)] \neq []$$

        Concluímos, então, que não é possível provar a existência de um
        membro do CA alpinista em KB'

        \item [d)] Use resolução com extração de resposta para descobrir quem é o membro do Clube
        Alpino que é alpinista mas não esquiador.

        $$[\neg CA(x), Esquiador(x), Alpinista(x)] (IV)$$
        $$[\neg CA(x), Gosta(x, neve), Alpinista(x)] (VI)$$

        Primeira tentativa falha:

        $$[CA (Antonio)] (I)$$

        $$\left[\frac{x}{Antonio}\right][Gosta(Antonio, neve), Alpinista(Antonio)]$$

        Segunda tentativa também falha:

        $$[CA (Joao)] (III)$$

        $$\left[\frac{x}{Joao}\right][Gosta(Joao, neve), Alpinista(Joao)]$$

        Terceira tentativa:

        $[CA (Maria)]$, $[Gosta(Antonio, neve)]$, $\left[\frac{x}{neve}\right][\neg Gosta (Antonio,x), \neg Gosta (Maria,x)]$  $(II, X, VII)$

        Temos, então:

        $$alpinista(Maria)$$

        Logo, Maria é alpinista.

    \end{itemize}


\end{enumerate}



\end{document}