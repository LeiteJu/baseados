\documentclass{article}
\usepackage[brazil]{babel}
\usepackage[utf8]{inputenc}
\usepackage{url}
\usepackage{multicol}
\usepackage{indentfirst}
\usepackage[bottom=2cm,top=2cm,left=2cm,right=2cm]{geometry}

\title{Sistemas Baseados em Conhecimento\\
    \large \texttt{[MAC0444]}\\
    \large Renata Wassermann}

\author{Julia Leite}

\begin{document}
    
\maketitle

\tableofcontents

\section{Introdução}

\textbf{Teste de Turing}: para um computador ser considerado inteligente, ele
teria que enganar um examinador (achar que seria humano)

\textbf{Habilidades envolvidas}

\begin{itemize}
    \item [-] Processamento de Linguagem Natural
           
    \item [-] Representação do Conhecimento
        Codificar coisas conhecidas (representar com símbolos) $\rightarrow$ 
        manipular símbolos que codificam proposições para produzir reprentação de 
        novas proposições 
        $$1011 + 10 \rightarrow 1101$$
    
    \item [-] Aprendizado de Máquina
\end{itemize}

Ideia de bom senso: \textit{Na sala, Lisa pegou o jornal e caminhou até a cozinha. Onde está o jornal?}\\
Considerar: representação (cenário) e raciocínio (inferências sobre o cenário)\\
Formalização: axiomas

\subsection{Sistema baseado em conhecimento}

\textbf{Sistema baseado em conhecimento}: deve ter estruturas que:

\begin{itemize}
    \item [-] podem ser interpretadas de forma proposicional
    \item [-] determinam o comportamento do sistema
\end{itemize}

Hoje, a informação é a semântica atribuída ao dado e o conhecimento 
é o depende do propósito atribuído... Adquirir conhecimento é fazer a
modelagem de domínio


\section{Lógica de Primeira Ordem}

\subsection{Lógica}

\textbf{Lógica} é uma linguagem formal bem conhecida e desenvolvida
Tem aparato para raciocínio (calcular consequências)

\subsection{Conhecimento ontológico}

Separando a parte da representação do conhecimento:

$$KR = Logica + Ontologia + Computacao$$

\begin{itemize}
    \item [-] Lógica: estrutura formal e regras de inferência
    \item [-] Ontologia: estudo do domínio
    \item [-] Computação: pra sair da discussão filosófica
\end{itemize}

\subsection{The Naïve Physics Manifesto (Hayes, 1978, 1983)}

\begin{itemize}
    \item [-] Proposta de desenvolver uma teoria formal englobando todo o
    conhecimento de física de leigos
    \item [-] Conhecimento na forma declarativa
    \item [-] Teoria organizada em clusters de conceitos e axiomas
\end{itemize}

\subsection{CYC}

\begin{itemize}
    \item [-] Projeto iniciado em 1984 por Doug Lenat
    \item [-] Formalizar conhecimento necess´ario para racioc´ınio
    envolvendo bom senso
    \item [-] Ideia: formalizar micromundos
    \item [-] Uso não é trivial
\end{itemize}


\end{document}